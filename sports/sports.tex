%%%%%%%%%%%%%%%%%%%%%%%%%%%%%%%%%%%%%%%%%
% "ModernCV" CV and Cover Letter
% LaTeX Template
% Version 1.11 (19/6/14)
%
% This template has been downloaded from:
% http://www.LaTeXTemplates.com
%
% Original author:
% Xavier Danaux (xdanaux@gmail.com)
%
% License:
% CC BY-NC-SA 3.0 (http://creativecommons.org/licenses/by-nc-sa/3.0/)
%
% Important note:
% This template requires the moderncv.cls and .sty files to be in the same 
% directory as this .tex file. These files provide the resume style and themes 
% used for structuring the document.
%
%%%%%%%%%%%%%%%%%%%%%%%%%%%%%%%%%%%%%%%%%

%----------------------------------------------------------------------------------------
%	PACKAGES AND OTHER DOCUMENT CONFIGURATIONS
%----------------------------------------------------------------------------------------

\documentclass[11pt,a4paper,sans]{moderncv} % Font sizes: 10, 11, or 12; paper sizes: a4paper, letterpaper, a5paper, legalpaper, executivepaper or landscape; font families: sans or roman

\moderncvstyle{casual} % CV theme - options include: 'casual' (default), 'classic', 'oldstyle' and 'banking'
\moderncvcolor{blue} % CV color - options include: 'blue' (default), 'orange', 'green', 'red', 'purple', 'grey' and 'black'

\usepackage{lipsum} % Used for inserting dummy 'Lorem ipsum' text into the template

\usepackage[scale=0.75]{geometry} % Reduce document margins
%\setlength{\hintscolumnwidth}{3cm} % Uncomment to change the width of the dates column
%\setlength{\makecvtitlenamewidth}{10cm} % For the 'classic' style, uncomment to adjust the width of the space allocated to your name

%----------------------------------------------------------------------------------------
%	NAME AND CONTACT INFORMATION SECTION
%----------------------------------------------------------------------------------------

\firstname{Elte} % Your first name
\familyname{Hupkes} % Your last name

% All information in this block is optional, comment out any lines you don't need
\title{Curriculum Vitae}
\address{Beukenplein 17-1}{1092 BA, Amsterdam}
\mobile{+31 (0)6 4007 2392}
\email{elte@hupkes.org}
\homepage{http://metaphoric.nl}{metaphoric.nl} % The first argument is the url for the clickable link, the second argument is the url displayed in the template - this allows special characters to be displayed such as the tilde in this example
\photo[70pt][0.4pt]{../elte.jpg} % The first bracket is the picture height, the second is the thickness of the frame around the picture (0pt for no frame)
\quote{Een Curriculum Vitae gericht op mijn algemene en 
sport- / fitnessgerelateerde activiteiten.}

%----------------------------------------------------------------------------------------

\begin{document}

\makecvtitle % Print the CV title

%----------------------------------------------------------------------------------------
%	EDUCATION SECTION
%----------------------------------------------------------------------------------------

\section{Educatie}
\cventry{2014--heden}{Fitness Trainer Level 1}{\textsc{AALO}}{Amsterdam}{}{}
\cventry{2013--heden}{Masters of Computational Science}{Universiteit van Amsterdam}{}{}{}
\cventry{2008--2010, 2011-2012}{Bachelor Informatica}
{Universiteit van Amsterdam}{}{Cum Laude}{}  % Arguments not required can be left empty

\section{Werk}
\cventry{2010--heden}{Hoofdontwikkelaar}{\textsc{SRXP}}{Amsterdam}{}{Bezig met
het ontwikkelen en uitgeven van software ter vereenvoudiging van
onkostendeclaraties.}
\cventry{2008--2010}{Ontwikkelaar}{\textsc{Fotoalbum.nl}}{Amsterdam}{}{
Ontwikkeling en beheer van consumentensoftware voor fotoalbums.}
\cventry{2007--2008}{Developer}{\textsc{Mercurius Event Solutions}}{Wormerveer}{}{Ontwikkeling en uitrol van software voor evenementen.}

\section{Ervaring}
\cventry{2012--2013}{Voorzitter}{Yamakasi}{Amsterdam}{}{Voorzitter van
studenten-sportdispuut Yamakasi.}
\cventry{2012--2013}{"Sportbaas"}{Yamakasi}{}{}{
Sportorganisator voor studenten-sportdispuut Yamakasi. Dit omvatte het
organiseren van een wekelijks verschillende sport-activiteit voor de +-
15 actieve leden, zowel als het met regelmaat leiden van een bootcamp-achtige
training.}
\cventry{2012--2013}{Voetbalcoach}{Team D.E.R.M}{}{}{Coaching activiteiten
voor een dames-zaalvoetbalteam.}

\section{Sportgeschiedenis}
\cventry{2012--heden}{Obstacle Course Running (OCR)}{}{}{}{Actief deelnemer
aan obstakelruns, als eerste over de finish in verscheidene races
waaronder Mud Masters Biddinghuizen in 2014.}
\cventry{2007--heden}{Fitness}{}{}{}{Training gericht op algehele fitness,
zowel cardiovasculair als kracht, met als doel verscheidene sporten
optimaal te kunnen beoefenen.}
\cventry{2007--heden}{Hardlopen}{}{}{}{Hardlooptraining voornamelijk
gericht op de midden- tot lange afstand. Gefinisht in een marathon
en verscheidene halve marathons.}
\cventry{2000--heden}{Schaatsen}{}{}{}{Door de jaren heen schaatstraining
op verschillende trainings-/comptetitieniveaus.}
\cventry{1996--2013}{Voetbal}{}{}{}{Actieve speler in de selectie
van \textsc{TABA} en \textsc{RVW}.}
\cventry{1995--heden}{Fietsen}{}{}{}{In vrijwel iedere vakantie een lange
fietstocht ondernomen, naar bijvoorbeeld Spanje, Zwitserland of Itali\"{e}.}

\section{Talen}
\cvitemwithcomment{Nederlands}{Moedertaal}{}
\cvitemwithcomment{Engels}{Vloeiend in zowel spreken als schrijven}{}

\section{Interesses}

\renewcommand{\listitemsymbol}{-~} % Changes the symbol used for lists
\cvlistdoubleitem{Obstacle Course Running}{Hardlopen}
\cvlistdoubleitem{Fitness}{Schaatsen}
\cvlistitem{Voetbal}
\cvlistdoubleitem{Gitaar}{Zang}
\cvlistitem{Piano}
\cvlistdoubleitem{Techniek}{Wetenschap}

\section{Motivatie}
Sport speelt al sinds ik mij kan herinneren een belangrijke rol in mijn
leven. Van de vele vakanties op de fiets tot de weekends gevuld met
voetbalwedstrijden in verschillende teams - er is geen periode geweest
waarin ik mij niet met sport heb bezig gehouden. Wie veel sport krijgt onvermijdelijk met blessures te maken. Deze waren voor mij nooit 
van grote invloed,
tot ik in maart 2013 met een overbelaste knie een voetbalwedstrijd speelde
en mijn voorste kruisband scheurde. In de revalidatie die daarop volgde
heeft mijn interesse zich verbreedt, door de realisatie dat je
door op een logische manier naar je lichaam 
te kijken en te trainen veel meer uit
sport kunt halen. Waar ik altijd al fanatiek bezig was met het beoefenen
van sport, ben ik nu ook enthousiast gaan kijken naar het op verstandige
wijze sterker maken van je lichaam met als doel optimaal deel te kunnen
nemen aan de sport die je leuk vindt. Om deze interesse meer
concreet te maken ben ik in de herfst van 2014 begonnen met een opleiding
tot fitnessinstructeur, met als doel dit in de toekomst uit te breiden
tot een certificatie als personal trainer. Ik hoop op die manier mijn
kennis van - en enthousiasme over sport op andere mensen te kunnen overdragen.

\end{document}